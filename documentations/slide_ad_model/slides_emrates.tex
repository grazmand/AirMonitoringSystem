\documentclass{beamer}
\usetheme{Boadilla}
\usepackage{graphicx}
\usepackage{amsmath}
\usepackage{mathtools}

\graphicspath{{./images/}}
\setbeamertemplate{caption}[numbered]

\title{Air Quality Monitoring Wireless System}
%\subtitle{Basics}
\author{Graziano A. Manduzio, PhD student}
\institute{University of Florence}
\date{\today}

%% new command
\newcommand{\B}{\pmb{B}}
\newcommand{\mb}{\mathbf}
\newcommand{\be}{\begin{equation}}
\newcommand{\ee}{\end{equation}}
\newcommand{\bs}{\boldsymbol}
\newcommand{\bff}{\begin{frame}}
\newcommand{\eff}{\end{frame}}

\begin{document}

%%%%%%%%%%%%%%%%%%%%%%%%%%%%%%%%%%%%%%%%%%%%%%%%%%%%%%%%%%%%%%%%%%%%
\begin{frame}
	\titlepage
	\center{supervisors: Giorgio Battistelli, Luigi Chisci ,Nicola Forti and Roberto Sabatini}
\end{frame}

%%%%%%%%%%%%%%%%%%%%%%%%%%%%%%%%%%%%%%%%%%%%%%%%%%%%%%%%%%%%%%%%%%%%
\begin{frame}
Advection-diffusion-reaction model equation: second order pde equation
\begin{equation}
		\displaystyle{\frac{\partial x}{\partial t}} - \lambda \nabla^2 x + \mb{v}^T \nabla x  ~=~ f \,\,\,\,\, \mbox{in } \mathbb{R}^2
\end{equation}

where:\\
	$x(\mb{p},t)$ is the space-time dependent pollutant concentration field, defined over the space-time domain with an initial condition $x(\mb{p},0)=x_{0}$;\\
	$\mb{p} \in \mathbb{R}^2$ denotes the $2$-dimensional position vector;\\
	$t \in \mathbb{R}^{+}$ denotes time;\\
	$\lambda$ is the diffusion coefficient;\\
	$\mb{v}(\mb{p},t)$ is the advection velocity vector;\\
	$f(\mb{p},t)$ represents the internal sources of pollution.\\
	$f(\mb{p},t)=s(t)F(\mb{p})$
\end{frame}
%%%%%%%%%%%%%%%%%%%%%%%%%%%%%%%%%%%%%%%%%%%%%%%%%%%%%%%%%%%%%%%%%%%%%%
\begin{frame}
We need to truncate the physical domain to solve numerically the problem by using the finite element method (FEM).
\end{frame}

%%%%%%%%%%%%%%%%%%%%%%%%%%%%%%%%%%%%%%%%%%%%%%%%%%%%%%%%%%%%%%%%%%%
\begin{frame}
variational formulation of the problem:\\
\[
	\int_\Omega \frac{\partial x}{\partial t} \varphi \, d\mb{p} \, - \,
	\lambda \int_\Omega ~\nabla^2 x~ \varphi \, d\mb{p} +
	\int_\Omega \mb{v}^T  ~\nabla x~ \varphi \, d\mb{p} =
	\int_\Omega f \varphi \, d\mb{p}
	\]
	where $\varphi(\mb{p})$ is a generic space-dependent weight function.
\\ 	
$\varphi \nabla^2 x = \nabla . (\varphi \nabla x) - \nabla \varphi . \nabla x$\\

\[
	\int_\Omega \frac{\partial x}{\partial t} \varphi \, d\mb{p} \, - \,
	\lambda \int_\Omega \nabla . (\varphi \nabla x) \, d\mb{p} + \lambda \int_\Omega \nabla \varphi . \nabla x \, d\mb{p} +
	\int_\Omega \mb{v}^T  ~\nabla x~ \varphi \, d\mb{p} =
	\int_\Omega f \varphi \, d\mb{p}
	\]\\
	
	\[
	\begin{array}{l}
	\displaystyle
	\int_\Omega \frac{\partial x}{\partial t} \varphi \, d\mb{p} \, - \,
	\lambda \int_{\partial \Omega} \varphi \nabla x . \mb{n} \, d\mb{p} + \lambda \int_\Omega \nabla \varphi . \nabla x \, d\mb{p} +
	\int_\Omega \mb{v}^T  ~\nabla x~ \varphi \, d\mb{p} = \\ [4mm]
	\displaystyle
	s(t) \int_\Omega F(\mb{p}) \varphi \, d\mb{p}
	
	\end{array}
	\]
	
	
\vspace{0.1cm}
FEM approximation:\\
\begin{center}

$	x(\mb{p},t) = \sum_{j=1}^{n} \phi_{j}(\mb{p}) \, x_j(t) ~=~ \boldsymbol{\phi}^T(\mb{p}) \, \mb{x}(t)$

\end{center}
\vspace{0.1cm}

\end{frame}

\begin{frame}

FEM derivative model:


\begin{multline} 
\small{\underbrace{\left[ \int_\Omega  
			\bs{\phi}(\mb{p}) \,\bs{\phi}^T(\mb{p}) d\mb{p} 
			\right]}_{\mb{M}} \dot{\mb{x}}(t)  + 
		\underbrace{\left[ \lambda \int_\Omega 
			\nabla	\bs{\phi}(\mb{p}) \,\nabla \bs{\phi}^T(\mb{p}) d\mb{p} 
			\right]}_{\mb{S}_\lambda} \mb{x}(t) }\\
\small{+ \underbrace{\left[ \int_\Omega 
			\bs{\phi}(\mb{p}) ~ \mb{v}^T(\mb{p}) \,\nabla \bs{\phi}^T(\mb{p}) \, d\mb{p} 
			\right]}_{\mb{G}} \mb{x}(t) }
	\small{+ \underbrace{\left[ \lambda \int_{\partial \Omega} 
			\beta(\mb{p}) \, \bs{\phi}(\mb{p}) \,\bs{\phi}^T(\mb{p}) \, d\mb{p}
			\right]}_{\mb{Q}_\beta} \mb{x}(t) = }\\
	=  \small{\underbrace{\left[\displaystyle{\int_\Omega} \bs{\phi}(\mb{p})
			F(\mb{p}) d\mb{p}	\right]}_{\mb{Q}_f} {s}(t) }
\end{multline} 
\end{frame}
%%%%%%%%%%%%%%%%%%%%%%%%%%%%%%%%%%%%%%%%%%%%%%%%%%%%%%%%%%%%%%%%%%%
\begin{frame}
example1: \\
$F(\mb{p})=\delta(\mb{p}-\mb{p}_s)$\\
$$\int_\Omega \bs{\phi}(\mb{p})
			F(\mb{p}) d\mb{p} = \int_\Omega \bs{\phi}(\mb{p})
			\delta(\mb{p}-\mb{p}_s) d\mb{p}=\bs{\phi}(\mb{p}_s)$$
example2: multiple point source\\
$f(\mb{p},t)=\sum_{s=1}^{N_s} s_s(t) \delta(\mb{p}-\mb{p}_s)$\\
$$\int_\Omega \bs{\phi}(\mb{p})
			F(\mb{p}) d\mb{p} = \sum_{s=1}^{N_s} \int_\Omega \bs{\phi}(\mb{p})
			\delta(\mb{p}-\mb{p}_s) d\mb{p}= \dfrac{1}{N_s} \sum_{s=1}^{N_s} \bs{\phi}(\mb{p}_s)$$
\end{frame}
%%%%%%%%%%%%%%%%%%%%%%%%%%%%%%%%%%%%%%%%%%%%%%%%%%%%%%%%%%%%%%%%%%%
\begin{frame}
Advection model:\\
advection velocity vector $=[5 5]$ m/sec.
\end{frame}
%%%%%%%%%%%%%%%%%%%%%%%%%%%%%%%%%%%%%%%%%%%%%%%%%%%%%%%%%%%%%%%%%%%
\begin{frame}
Diffusion model:\\
diffusion coefficient $=5$ ppm/m$^2$
\end{frame}
%%%%%%%%%%%%%%%%%%%%%%%%%%%%%%%%%%%%%%%%%%%%%%%%%%%%%%%%%%%%%%%%%%%
\end{document}



	

	
